
\documentclass[12pt]{article}

\title{Github Details}

\begin{document}
\maketitle

\section{Cloning Github Repository To Local Machine}
\begin{enumerate}
  \item Open your file explorer in Ubuntu or any other similar Linux variant to the location        you would like to save your repository to.
  \item Right click and select ``open terminal here'' to open the terminal in this location.
  \item If you have never used github before, make sure to run the following commands: \\
  \framebox{\parbox{\textwidth}{git config -{}-global user.name                       Your\textunderscore Full\textunderscore Name \\
                    git config -{}-global user.email youremail@wit                     s.ac.za}}
\\
  where Your\textunderscore Full\textunderscore Name and youremail@wits.ac.za will be replaced with your own details.
  \item Now type the following: \\
  \framebox{git clone https:/github.com/jasonstuart/SE\textunderscore Project\textunderscore 2017.git}\\
  This will clone the current repository to your machine.
  
  \item At this point you can navigate into the created folder (which houses the repository). Currently, you will see a subdirectory called ``Documents'' which is where all our remaining documentation will be located.

\end{enumerate}

\section{Editing The Repository}
When a new feature is going to be implemented, one must make sure to create a new branch for this feature and then add commits there until the feature is ready for deployment.

\subsection{Creating a new branch}
\begin{enumerate}
    \item In your terminal, type the following command to create a new branch\\
    \framebox{git branch BranchName}\\
    where BranchName can be any identifying name you would like.
    \item Change over to this branch now using this command:\\
    \framebox{git checkout BranchName}\\
    where BranchName is the name of your Branch you created.
\end{enumerate}

\subsection{Adding Commits}
\begin{enumerate}
    \item Once on the respective branch, and you have made modifications worthy of a commit, run the following code to add files to staging area.\\
    \framebox{git add fileName}\\
    with fileName being the name of the file modified
    \item Now commit the changes using:\\
    \framebox{git commit -m ``YourMessageHere''}
    \item Make sure to push this commit to the Github Repository using:\\
    \framebox{git push}\\
    and enter your github username and password when requested.
\end{enumerate}

\subsection{Merging Branch With Master Branch}
Once you are ready to merge your branch with the master branch, you can then merge the branches using:\\
\framebox{git merge yourBranch}\\
making sure that you are currently in the master branch.

\section{Accessing Documentation}
Please note that all documentation will be contained in the Document Subfolder on the Master Branch

\end{document}